
\documentclass{beamer}
%\usepackage{beamerarticle}
\usepackage{fancyhdr}
\usepackage{amsmath, amscd}
%\usepackage{enumerate}
\usepackage{amsthm}
\usepackage{amssymb}
%\usepackage[pdftex]{graphicx}
\usepackage{verbatim}
\usepackage[all]{xy}
%\usepackage{hyperref}
%\usepackage{homework}
%\usepackage[top=1.5in, bottom=1in, left=1.25in, right=1.25in]{geometry}

\def\baselinestretch{1} % not quite double spaced


\input{../../../mydefns/gsDefns}

\mode<presentation>
{
  \usetheme{default}
  \usecolortheme[RGB={46,100,115}]{structure}
  \usecolortheme{crane}
  \usecolortheme{orchid}
  \usecolortheme{whale}
  \useoutertheme{infolines}
}


\usepackage[english]{babel}


\usepackage[latin1]{inputenc}


\usepackage{times}

\title[The smallest root of a growth polynomial]{The smallest root of a growth polynomial}


\author[Sisodia]{Gautam Sisodia} 
\date{December 10, 2014}

\subject{Talks}

\begin{document}

\begin{frame}
\titlepage

\end{frame}


\begin{frame}
\frametitle{Growth polynomials}

A \textcolor{red}{growth polynomial} is a polynomial of the form
$$
f(t) = 1 - c_1 t - c_2 t^2 - \cdots - c_n t^n
$$

\pause

where the $c_i$'s
\begin{itemize}
\item are non-negative integers,
\pause
\item are relatively prime, and
\pause
\item have sum at least two. 
\end{itemize}

\pause

\hfil

\begin{ex}
\begin{itemize}
\item $1-2t$
\pause
\item $1-t-t^2$ 
\pause
\item $1-t^3-7t^4-t^8-2t^{10}$
\end{itemize}
\end{ex}

\end{frame}

\begin{frame}
\frametitle{The roots of $1-t-t^2$}

$1-t-t^2$ has roots $\frac{-1 \pm \sqrt{5}}{2}$

\pause

\begin{figure}[h]
\centering
\makebox[\textwidth][c]{\includegraphics[width = \textwidth]{11}}
\end{figure}


\end{frame}

\begin{frame}
\frametitle{The roots of $1-t-t^2-t^3$}

\begin{figure}[h]
\centering
\makebox[\textwidth][c]{\includegraphics[width = 0.35\textwidth]{111}}
\end{figure}


\end{frame}

\begin{frame}
\frametitle{The roots of $1-t-2t^4-3t^7$}

\begin{figure}[h]
\centering
\makebox[\textwidth][c]{\includegraphics[width = 0.5\textwidth]{1002003}}
\end{figure}


\end{frame}

\begin{frame}
\frametitle{The roots of $1-t^3-7t^4-t^8-2t^{10}$}

\begin{figure}[h]
\centering
\makebox[\textwidth][c]{\includegraphics[width = 0.5\textwidth]{0017000102}}
\end{figure}


\end{frame}




\begin{frame}
\frametitle{Conjecture}

Every growth polynomial has a simple positive real root that is smallest in modulus amongst the roots of the polynomial.

\pause

\hfil

\textcolor{blue}{
the computational skills to generate a massive amount of evidence, and the 
intuition to conjecture based on that evidence
}


\end{frame}


\begin{frame}
\frametitle{A similar result in the literature}

Dr. Solomyak points me to

\begin{thm}[Perron-Frobenius]
If $B$ is a primitive matrix, then $B$ has a simple positive eigenvalue that is largest in modulus amongst the eigenvalues of $B$.  
\end{thm}

\pause


\hfil

\textcolor{blue}{an eagerness to seek advice from mentors and peers}

\pause

\hfil

The eigenvalues of $B$ are exactly the roots of the characteristic polynomial of $B$.

\pause

\hfil

growth polynomial $\leftrightarrow$ characteristic polynomial?

\end{frame}

\begin{frame}
\frametitle{Directed graphs}

from $f(t) = 1-t-t^2-t^3$, draw $G = $

$$
\xymatrix@R=30pt@C=45pt{
b \ar@/^15pt/@<-0.5ex>[dr] & & c \ar@/^15pt/[dd] \\
& a \ar@/^15pt/[ul] \ar@/^15pt/@<-0.5ex>[ur] \ar@(l,d)[] \\
& & d \ar@/^15pt/[ul]
}
$$

\pause

which has \textcolor{red}{incidence matrix} $M = $
\begin{center}
\begin{tabular}{c|cccc}
& $a$ & $b$ & $c$ & $d$ \\
\hline
$a$ & 1 & 1 & 1 & 0 \\
$b$ & 1 & 0 & 0 & 0 \\
$c$ & 0 & 0 & 0 & 1 \\
$d$ & 1 & 0 & 0 & 0
\end{tabular}
\end{center}

\end{frame}

\begin{frame}

The characteristic polynomial of $M$ is

\begin{align*}
\det(tI - M) &= t^4 - t^3 - t^2 - t \\
&= t^4 f(1/t)
\end{align*}

\pause

\hfil

$\implies$
$$
r \text{ is a nonzero eigenvalue of } M \iff 1/r \text{ is a root of } f(t)
$$

\end{frame}

\begin{frame}
\frametitle{}
$$
M^4 = \pma{
7 & 4 & 4 & 2 \\
4 & 2 & 2 & 1 \\
2 & 1 & 1 & 1 \\
4 & 2 & 2 & 1
}
$$

i.e. $M$ is \tcr{primitive}

\pause

so Perron-Frobenius applies:
\begin{itemize}
\item $M$ has a simple positive eigenvalue that is largest in modulus amongst the eigenvalues of $M$, so
\item $f$ has a simple positive root that is smallest in modulus amongst the roots of $f$
\end{itemize}

\end{frame}

\begin{frame}
\frametitle{The idea}
$$
\xymatrix{
&*+<10pt>[F-:<10pt>]\txt{Graph theory \\ \tcr{incidence matrices}} \ar@/^15pt/@{<->}[rd] & \\
*+<10pt>[F-:<10pt>]\txt{Algebra \\ \tcr{growth polynomials}} \ar@/_15pt/@{<-->}[rr] \ar@/^15pt/@{<->}[ur] & & *+<10pt>[F-:<10pt>]\txt{Matrix analysis \\ \tcr{Perron-Frobenius}}
}
$$

\pause

\hfil
\hfil

\textcolor{blue}{the ability to connect insights from different disciplines}

\end{frame}

\begin{frame}
\frametitle{The result}

\begin{thm}[Sisodia]
Every growth polynomial has a simple positive real root that is smallest in modulus amongst the roots of the polynomial.
\end{thm}

\pause

\hfil

\textcolor{blue}{the commitment to see an idea through to completion}


\end{frame}


\begin{frame}
\frametitle{Thank you!}

\hfil

C. Holdaway and G. Sisodia. \tcb{Category equivalences involving graded modules over weighted path algebras and monomial algebras}. {\it Journal of algebra}, 405:75-91, 2014. \tcr{\href{http://arxiv.org/abs/1309.3352}{arXiv:1309.3352}}.

\hfil

G. Sisodia and S. P. Smith. \tcb{The Grothendieck groups of non-commutative non-noetherian analogues of $\mathbb{P}^1$ and regular algebras of global dimension two}. {\it Journal of algebra}, to appear. \tcr{\href{http://arxiv.org/abs/1403.0640}{arXiv:1403.0640}}.


\end{frame}

\end{document}

