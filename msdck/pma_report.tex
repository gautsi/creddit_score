


\documentclass[11pt]{amsart}

\usepackage{fancyhdr}
\usepackage{amsmath, amscd}
\usepackage{enumerate}
\usepackage{amsthm}
\usepackage{amssymb}
%\usepackage[pdftex]{graphicx}
\usepackage{verbatim}
\usepackage[all]{xy}
\usepackage{booktabs}
%\usepackage{hyperref}
%\usepackage{homework}
%\usepackage[top=1.5in, bottom=1in, left=1.25in, right=1.25in]{geometry}

\input{../../mydefns/gsDefns}


\begin{document}

\title[]{}
\date{\today}
\author{Gautam Sisodia} 
\address{ Department of Mathematics, Box 354350, Univ. Washington, Seattle, WA 98195}
\email{gautas@math.washington.edu}
\begin{abstract}
 blah, blah
\end{abstract}
\maketitle


\begin{thebibliography}{4}
%\bibitem{ass} I. Assem, D. Simson, and A. Skowronski, {\it Elements of the Representation Theory of Associative Algebras: Volume 1.}

%\bibitem{cb} W. Crawley-Boevey, {\it Lectures on Representations of Quivers.}

%\bibitem{westbury} B.W. Westbury, {\it Sextonions and the magic square.}

%\bibitem{mozrein} S. Mozgovoy and M. Reineke, {\it On the noncommutative Donaldson-Thomas invariants arising from brane tilings.}%, \href{http://arxiv.org/abs/hep-th/0207027v1}{http://arxiv.org/abs/hep-th/0207027v1}

%\bibitem{ishii2} A. Ishii and K. Ueda, {\it Dimer models and the special McKay correspondence.}%, \href{http://arxiv.org/abs/hep-th/0207027v1}{http://arxiv.org/abs/hep-th/0207027v1}


%\bibitem{hand} Lidia Angeleri H\"ugel, Henning Krause, and Dieter Happel (eds.), {\it Handbook of Tilting Theory}, London Mathematical Society Lecture Notes Series 332, Cambridge University Press, 2007.



\end{thebibliography}

 
\end{document}
